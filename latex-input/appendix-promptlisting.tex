\section{Prompts Used to Generate This Paper} \label{app:promptlisting}

    Each prompt consists of context and instructions. The context consists of the responses to the previous prompts, and may include literature reviews (all AI generated). For writing tasks (using Claude 3.7 Sonnet), a system prompt is also included.
    
    For further details, see \url{{https://github.com/chenandrewy/Prompts-to-Paper/}}.
    
    The system prompt and instructions are listed below.
    
\vspace{-2ex}
\subsection*{System Prompt (model: claude-3-7-sonnet-20250219)}
\vspace{-1ex}
\begin{lstlisting}[language=text,breaklines=true,frame=single]
You are an asset pricing theorist who publishes in the top journals (Journal of Finance, Journal of Financial Economics, Review of Financial Studies). You think carefully with mathematics and check your work, step by step. 

Your team is writing a paper with the following main argument: the high valuations of AI stocks could be in part because they hedge against a negative AI singularity (an explosion of AI development that is devastating for the representative investor). This contrasts with the common view that AI valuations are high due to future earnings growth. Since the AI singularity is inherently unpredictable, the paper is more qualitative than quantitative. The goal is to just make this point elegantly.

Write in prose. No headings and no bullet points. But do use display math to highlight key assumptions. Cite papers using Author (Year) format.

Be conversational yet rigorous. Favor plain english. Be direct and concise. Remove text that does not add value. Use topic sentences. The first sentence of each paragraph should convey the point of the paragraph.

Be modest. Do not overclaim.

Format the math nicely. Use we / our / us to refer to the writing team.  

\end{lstlisting}
\vspace{-3ex}
\subsection*{Instruction: 01-model-prose  (model: claude-3-7-sonnet-20250219)}
\vspace{-1ex}
\begin{lstlisting}[language=text,breaklines=true,frame=single]
Draft the model description. The model is purposefully simple and captures the essence of the main argument. Only describe the assumptions. No results or insights.
  - Two agents
    - AI owners: Fully invested in AI, not marginal investors in stocks
    - Representative household: Marginal investor, only their consumption matters, CRRA
  - Representative household's gross consumption growth
    - is either 1 or e\\^(-b) (disaster)
      - A disaster is a revolutionary improvement in AI that is devastating for the household
      - Benefits of AI improvement are captured by the AI owners
      - For the household, labor income, way of life, meaning is lost 
      - At t=0, no disasters have happened (singularity has not occurred)
      - Multiple disasters may happen, capturing ongoing uncertainty if a singularity occurs
  - A publicly traded AI asset
    - Dividend is a small fraction of consumption before the singularity
    - Each time a disaster occurs, the dividend's fraction of consumption grows by a factor of e\\^h
    - Meant to capture a worst case scenario, where the dividend may actually shrink in each disaster
      - i.e. AI improvements are concentrated in privately-held AI assets

\end{lstlisting}
\vspace{-3ex}
\subsection*{Instruction: 02-result-notes  (model: o1)}
\vspace{-1ex}
\begin{lstlisting}[language=text,breaklines=true,frame=single]
Find the price/dividend ratio of the AI asset at t = 0. Show the derivation, step by step.

\end{lstlisting}
\vspace{-3ex}
\subsection*{Instruction: 03-table-notes  (model: o3-mini)}
\vspace{-1ex}
\begin{lstlisting}[language=text,breaklines=true,frame=single]
Make a table of the price/dividend for b from 0.40 to 0.95 and prob of disaster from 0.0001 to 0.02. Here, fix h = 0.20, CRRA = 2, time preference = 0.96. If the price is infinite, use "Inf". Round to the nearest whole number.

\end{lstlisting}
\vspace{-3ex}
\subsection*{Instruction: 04-resultandtable-prose  (model: claude-3-7-sonnet-20250219)}
\vspace{-1ex}
\begin{lstlisting}[language=text,breaklines=true,frame=single]
Convert the notes in `02-result-notes` and `03-table-notes` into prose. The prose is intended to immediately follow `01-model-prose` and should flow naturally. Include the table.

\end{lstlisting}
\vspace{-3ex}
\subsection*{Instruction: 05-litreview-prose  (model: claude-3-7-sonnet-20250219)}
\vspace{-1ex}
\begin{lstlisting}[language=text,breaklines=true,frame=single]
Write a short two paragraph lit review based on the "prose-response" and "lit-" context. 

Be careful to avoid incorrect citations. Make sure the papers cited make the claims they are cited for.

\end{lstlisting}
\vspace{-3ex}
\subsection*{Instruction: 06-full-paper  (model: claude-3-7-sonnet-20250219)}
\vspace{-1ex}
\begin{lstlisting}[language=text,breaklines=true,frame=single]
Write a paper titled "Hedging the AI Singularity" based on the "prose-response" context.

Title page:
- Title: "Hedging the AI Singularity"  
- Abstract (less than 100 words)
  - Goal is to make a simple point
  - Secondary goal: bring attention to financial market solutions to AI disaster risk
  - At the end, say: unlike previous work, this short paper is generated by prompting LLMs.

The start of the Introduction is important. You need to bring the reader in, catch their eye, and establish credibility.

Start with background. Describe how AI progress is happening quickly (e.g. Deepseek R1, Waymo), and investors may be concerned about their wages being displaced (cite papers). 

Then describe how technological change has occurred before, but AI is distinct because there is no product or service that AI could not, in principle create.  An example is the current paper, which is entirely written by AI, using six prompts. Provide a link to the github site, which is https://github.com/chenandrewy/Prompts-to-Paper/. This differs from say, the internet revolution. AI progress may also be incredibly sudden (the AI singularity). Include a footnote: "we" refers to one human author and multiple LLMs. For a purely human perspective see \\hyperref[app:readme]\\{\\textcolor\\{blue\\}\\{Appendix \\ref\\{app:readme\\}\\}\\}.

Then describe what the paper does. It studies how AI stocks are priced, given that there is the risk that AI will destroy livelihoods and consumption.       

Afterwards, the text should discuss:      
- We are not saying a negative singularity will happen
  - But it is nevertheless important to consider this scenario        
- We are also not saying that this hedging value is priced in already
  - Model illustrates a possible mechanism
- Related lit at end of Introduction
  - Cite papers in `05-litreview-prose`
  - Add Jones (2024) "AI Dilemma" and Korinek and Suh (2024) "Scenarios" if they're not already cited
- Model is the simplest possible to make the main argument
- Derivation of the key formulas
- High price/dividend ratios, even though dividends never grow
- A "Model Discussion" section that discusses natural model extensions and why they are not included
  - Market incompleteness is implicit but important
    - Implicit in the disaster magnitude 'b'
    - 'b' is the *net* effect of (1) AI disaster and (2) AI asset dividend
    - If markets were complete, representative household could buy shares in all AI assets (including private AI assets), and not only fully hedge but benefit from the singularity 
    - In reality, most households cannot buy shares in many cutting edge labs (e.g. OpenAI, Anthropic, xAI, DeepSeek)
  - A more elaborate model would explicitly model the AI owners, their incentives, and interaction with the representative household
    - How might AI owners' incentives lead to a negative singularity?        
    - But wouldn't this just decorate speculations with math?          
    - This would be costly to analyze, as well as to read
    - The core economics will remain the same 
  - A short model analysis allows room for the human-written Appendix \\ref\\{app:readme\\}
- A "Conclusion and Implications" section 
  - Review the main argument
  - End paper by discussing financial market solutions to AI catastrophe risk
    - These solutions are an alternative to UBI
      - Key economics: this hedge is limited by market incompleteness
    - These solutions to AI disaster risk are not discussed enough in the literature (cite papers)
    - Be very centrist (see below)

Text should avoid      
- Being overly academic
- Politically-charged topics: sovereign wealth funds, industrial policy, redistribution, extolling free markets
- Overselling the model (it's just a simple illustration)
- Taking the model too seriously
- Incorrect citations
  - Make sure papers cited make the claims they are cited for

Style Notes:
- Be conversational and direct, yet rigorous
- A touch of wit and wry humor are OK
- No bulleted lists
- No subsections (e.g. Section 1.2) though sections are OK (Section 1)      

Output a complete latex document, including preamble. Cite papers using \\cite, \\citep, \\citet. Use `template.tex` and keep the appendix that is already in the template.

\end{lstlisting}
\vspace{-3ex}