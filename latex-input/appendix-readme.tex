\section{A Purely Human Perspective} \label{app:readme}
The following is the README.md file from the GitHub repository:
\vspace{-1ex}
\begin{mdframed}[linewidth=1pt, linecolor=black]
\begingroup
\small
\ttfamily
\linespread{1}
\setlength{\parindent}{0pt}
\setlength{\parskip}{0.5\baselineskip}
\sloppy
\setlength{\emergencystretch}{3em}
\textbf{\textcolor{red!70!black}{\# Prompts-to-Paper}}

Writes a paper about hedging a negative AI singularity, using AI.

- \colorbox{gray!10}{\textcolor{red!70!black}{make-paper.py}} writes a paper

- \colorbox{gray!10}{\textcolor{red!70!black}{plan0403-streamlined.yaml}} contains the prompts

- \colorbox{gray!10}{\textcolor{red!70!black}{make-many-papers.py}} runs \colorbox{gray!10}{\textcolor{red!70!black}{make-paper.py}} many times.


The README is entirely human-written. Please forgive typos and errors.

\textbf{\textcolor{red!70!black}{\# Motivation}}

On March 8, 2025 I thought I should write a paper about hedging the AI singularity.

I was worked up. I had been repeatedly shocked by AI progress. I was using AI reasoning, vibe coding, and AI lit reviews in my daily life. Six months ago, I had thought each of these things is impossible.

What will happen in the next six years?! Will my entire job be replaced by AI? I have no idea.

But I do know that if there are huge disruptions, then tech stocks will benefit. So if anything bad happens to my human capital, I could at least partially hedge. Strangely, I hadn't heard about this concept before.

I asked a friend if he would be interested in working on this paper. Unfortunately, he was busy with revision deadlines for the next month. 

So, I thought I should use AI to write the paper. It would be an elegant way to make my point. It would also hint at where the research process is going in this strange age of AI.

This project was inspired by \href{https://www.nber.org/papers/w33363}{\textcolor{blue}{Novy-Marx and Velikov (2025)}}  and \href{https://arxiv.org/abs/2408.06292}{\textcolor{blue}{Chris Lu et al. (2024)}}. These projects show how AI could generate massive amounts of academic research. My goal differs in quality over quantity. I want to generate just one paper, but one paper that (I hope) people find is worth reading.

\textbf{\textcolor{red!70!black}{\# The Paper Generation Process}}

A natural question is: is this paper **really** written by AI? 

I’d say the AI are junior co-authors. 

If they were human, I would absolutely have to give o1, Sonnet, and ChatGPT Deep Research credit as co-authors. They did the math, writing, and literature reviews. Sonnet also wrote much of the code (via the Cursor AI IDE).

Of course, I did substantial hand-holding (see \colorbox{gray!10}{\textcolor{red!70!black}{plan0403-streamlined.yaml}}). Writing this paper would have been easier if I had done more of the work myself. But that can happen with human co-authors too. 

Perhaps in 12-24 months, AI can be equal co-authors. I wouldn’t be surprised if I could just ask Gemini 3.5 to "write a model to formalize my intuition," and it would get it right.

\textbf{\textcolor{red!70!black}{\#\# Paper Iterations}}

Like human-written papers, the writing process was iterative. The first formalizations were terrible. 

\colorbox{gray!10}{\textcolor{red!70!black}{plan0313-laborshare.yaml}} contains prompts for a neoclassical growth model, where the capital share suddenly increases. ChatGPT-o1 \href{https://chatgpt.com/share/67ee989a-50b4-800d-842f-ab71d2424c53}{\textcolor{blue}{patiently explained to me}} why this is a bad model.

me:
 > I thought there would be a wage risk effect that leads to higher investment for the more risk averse agent. High capital share means low or even no wage income.

ChatGPT-o1:
> Below is an explanation of why one might **expect** a "wage-risk" channel in which **more** risk aversion could lead to **more** *additional* investment (relative to the no-jump benchmark) in the event that the capital share might jump to 1. However, this channel **does not operate** in the usual **representative-agent** version of the model---there, wage and capital income ultimately go to the *same* agent, so there is no meaningful "hedge" of wage risk. Instead, the wage-risk hedge arises naturally if you depart from the pure representative-agent setting (for example, if households receive labor income but must *choose* how much capital they own).

The final \colorbox{gray!10}{\textcolor{red!70!black}{plan0403-streamlined.yaml}} uses a simplified Barro-Rietz disaster model, with two agents (though only one is relevant for stock prices). I went through several iterations of this model with Claude 3.7 Sonnet (thinking mode) and ChatGPT-o1. The only derivations I did myself were to check o1's work.

\textbf{\textcolor{red!70!black}{\#\# Literature Reviews}}

A key element was generating lit reviews (\colorbox{gray!10}{\textcolor{red!70!black}{./lit-context/}}) to give the AI context. I used ChatGPT's Deep Research (launched Feb 2025) until I ran out of credits. Claude Web Search (launched March 2025, after I began the project) did the remainder.

These new products were a game changer. Both \href{https://www.nber.org/papers/w33363}{\textcolor{blue}{Novy-Marx and Velikov (2025)}}  and \href{https://arxiv.org/abs/2408.06292}{\textcolor{blue}{Chris Lu et al. (2024)}} ran into hallucinated citations. OpenAI Deep Research and Claude Web Search had no problems if they were used with care. 

More broadly, knowing how to use which AI and when was helpful for generating a good paper. 

\textbf{\textcolor{red!70!black}{\#\# AI Model Selection }}

o1 did the theory, and sonnet thinking did the writing. It's well known that these are the strengths of these two models. 

Sonnet thinking is OK at economic theory. But I found that it was not assertive enough. It led me down wrong paths because it was too eager to come up with some ideas that for my story (even if they did not make sense).  

I briefly tried having Llama 3.1 470b do the writing. It was terrible! It would be extremely difficult to generate a paper worth reading that way. 

I did not try many other models, in order to get this paper out quickly. Gemini 2.5's release, at the end of March 2025, was *hype*. I tried it out briefly and was impressed. But I gritted my teeth and ignored it. I'd never get the paper finished if I wanted to really try to explore alternative models. 

\textbf{\textcolor{red!70!black}{\#\# Picking the best of N papers}}

The writing quality varies across each run of the code. There is both a good tail and a bad tail. Some drafts, I found quite insightful! Others, had flagrant errors in the economics. 

Rather than try to prompt engineer an error free, insightful paper, I decided to just generate N papers and choose the best one. 

Some papers had problematic cites (\colorbox{gray!10}{\textcolor{red!70!black}{run01}}). Others provided low-quality model discussions (\colorbox{gray!10}{\textcolor{red!70!black}{run02}}) or poor explanations of the algebra (\colorbox{gray!10}{\textcolor{red!70!black}{run03}})

\textbf{\textcolor{red!70!black}{\# Lessons about Research }}

A common response to \href{https://www.nber.org/papers/w33363}{\textcolor{blue}{Novy-Marx and Velikov (2025)}} is that "people are not ready for this." I heard concerns that peer review process will be inundated with AI-generated slop.

Working on this paper gave me a different perspective. It made me think about the fundamentals. I think the fundamentals are the following:

1. Readers want to learn something interesting and true.

2. Readers don't want to check all the math.

3. A system of author reputations makes 1 and 2 possible.

AI-generated papers don't change any of these fundamentals.  Critically, item 3 made me quite cautious about putting my name on AI slop. As a result, I don't think AI-generated papers will change much about peer review, at least not the current generation of AI.

\textbf{\textcolor{red!70!black}{\#\# Limitations of the Current AI (April 7, 2025)}}

This will likely be out of date by the time you read it.

But right now, AI is like a junior co-author with a talent for mathematics and elegant writing, but sub-par economics reasoning. 

For example, 3.7 Sonnet sometimes fails to recognize that the economic model does not capture an important channel. This is a common scenario in economics writing (no model can capture everything). The standard practice is to dance gingerly around the channel in the writing. A decent PhD student can recognize this. But Sonnet cannot. Instead, 3.7 Sonnet will write beautiful prose about the channel anyway, even though it's not really being studied properly. 

AI also cannot generate a satisfying economic model on their own (at least not satisfying to me). I tried asking o1 and Sonnet to generate a model to illustrate the point I'm trying to make. The resulting models were either too simplistic or did not lead to a clean analysis. They often introduced complications that I found unnecessary. 

There could be models with capabilities that I missed. But my sense is that ChatGPT-o1 and Claude 3.7 Sonnet are close to the best for producing economic research.

But more importantly, how long will these limitations last? 

\textbf{\textcolor{red!70!black}{\#\# The Future of AI and Economics Research}}

At some point, 2024-style economic analysis will be "on tap." You'll be able to go to a chatbot and ask "write me a paper about hedging AI disaster risk," and it will return you something like this paper (or perhaps something better). 

"Economics on tap" could be a disaster for the economics labor market. It would certainly mean that AI is an extremely cheap substitute for at least some economists' labor. It's possible that this would result in a strong substitution away from labor.

The optimistic argument is that AI also complements economists' labor. Perhaps, the number of economists will remain the same, but research output increases in terms of both quantity and quality. 

But I think there are reasons why total research output is limited. Two key factors in academic publishing are attention and reputation (\href{https://repub.eur.nl/pub/6875/2001-0221.pdf}{\textcolor{blue}{Klamer and van Dalen 2001, J of Economic Methodology}}). Readers can only pay attention to so many scholars. These scholars, in turn, can only pay attention to so may projects. 

I'm not saying that I *expect* a disaster for the economics labor market. But it's definitely a scenario that economists should think about. 
\endgroup
\end{mdframed}
\vspace{-3ex}