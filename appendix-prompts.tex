\appendix
\section{Prompts Used to Generate This Paper}
\begin{quote}
This appendix contains the prompts used to generate this paper. Each prompt was given to a large language model, which then generated the corresponding section of text. The prompts are presented in the order they were used.
\end{quote}
\vspace{2em}
\subsection*{Prompt 1: 01-model-prose}
\textbf{Model:} sonnet
\vspace{0.5em}
\begin{lstlisting}[language=text,breaklines=true,frame=single]
Draft the model description. Only describe the assumptions. No results or insights. Be modest. 

Assume the reader is an asset pricing expert and knows standard results like the SDF and the 1 = E(MR). 

Use the following outline:
  - The model is purposefully simple and captures the essence of the main argument
  - Two agents
    - AI owners
      - Fully invested in AI, not marginal investors
    - Representative household
      - Marginal investor: only their consumption matters
      - CRRA = \\gamma
  - Consumption growth
    - \\log \\Delta c\\{t+1\\} = 0 if no disaster
    - \\log \\Delta c\\{t+1\\} = -b if disaster (prob p)
      - in these disasters, the value of AI is captured by the AI owners. 
      - Labor income, way of life, meaning of household is lost 
      - at t=0, no disasters have happened (singularity has not occurred)
      - Multiple disasters may happen, capturing ongoing uncertainty if a singularity occurs
  - AI asset
    - Captures publicly traded AI stocks
    - Dividend D\\_t = a exp\\^\\{h N\\_t\\} C\\_t
      - a > 0 is small, AI stocks are currently a minor share of the economy
      - N\\_t is the number of disasters that have occurred up to and including time t
      - h > 0: each time a disaster occurs, the AI asset grows as a share of the economy

\end{lstlisting}
\vspace{1em}
\subsection*{Prompt 2: 02-result-notes}
\textbf{Model:} o1
\vspace{0.5em}
\begin{lstlisting}[language=text,breaklines=true,frame=single]
Find the price/dividend ratio and risk premium of the AI asset at t = 0. The risk premium is the expected return (including dividends) minus the risk-free rate.    Derive the formulas, step by step. 

Do not restate the assumptions.

Try to make the final formulas self-contained and not depend on the other final formulas.

\end{lstlisting}
\vspace{1em}
\subsection*{Prompt 3: 03-table-notes}
\textbf{Model:} o3-mini
\vspace{0.5em}
\begin{lstlisting}[language=text,breaklines=true,frame=single]
Illustrate the results in `02-result-notes` with a couple numerical examples. Focus on gamma = 2, beta = 0.96, and p = 0.01. What values of b and h lead to convergence of the price/dividend ratio? 

Then make a table of the price/dividend ratio at t=0 for b = 0.4, 0.6, 0.8, 0.95 and p = 0.0001, 0.001, 0.01, 0.02. Here, fix h = 0.2. If the price is infinite, use "Inf"

Make a table for the risk premium (expected return - risk-free rate) in percent (100*(gross return - 1)). If the price is infinite, leave the cell blank.

\end{lstlisting}
\vspace{1em}
\subsection*{Prompt 4: 04-price-div-prose}
\textbf{Model:} sonnet
\vspace{0.5em}
\begin{lstlisting}[language=text,breaklines=true,frame=single]
Convert the notes in `02-result-notes` and `03-table-notes` into prose. The prose is intended to follow`01-model-prose` and should flow naturally, ultimately to be in the same "Model" section.

The prose does not cover all results. It covers only the derivation and table for the price/dividend ratio.

The derivation should be concise and somewhat terse. Fix notational issues like the re-use of the same variable name for different quantities.

The table should be clean and simple. 

This is the key text of the paper. Combined with `01-model-prose`, it will be a section called "The Main Argument." Conclude the text by using the table to make the main argument. Discuss the intuition behind the table.

\end{lstlisting}
\vspace{1em}
\subsection*{Prompt 5: 05-discussion-prose}
\textbf{Model:} sonnet
\vspace{0.5em}
\begin{lstlisting}[language=text,breaklines=true,frame=single]
Write the "Model Discussion" section. Discuss the following subtleties of the model in prose (no math):

- On model interpretation:      
  - b should be considered the net result of (1) the AI disaster and (2) the dividend from the AI asset. In other words, the AI asset only partially hedges the disaster, and the key variable in the model is the net effect. This partial hedging is critical to the main argument.
  - The representative household cannot buy shares of all AI assets. Otherwise hedging would be complete.
    - This is consistent with reality. Most people can't buy shares of OpenAI, Anthropic, xAI, etc.
- On making the model more realistic:
  - Modeling the two components of b explicitly would require two Lucas trees, as in Cochrane et al. or Martin's Ecta. This adds state variables and significant complications.
  - Though the model is an endowment economy, one can think of the consumption path as the result of a production economy in which the labor share may drop sharply, and the share of output produced by privately-held AI increases sharply.
  - While the traditional literature would write down a richer, dynamic model and rigorously calibrate model parameters, in this case calibration is completely unreasonable, given the uncertainty about the singularity.    

Respond with only new content. Do not repeat anything in the context.    

\end{lstlisting}
\vspace{1em}
\subsection*{Prompt 6: 06-litreview-notes}
\textbf{Model:} sonnet
\vspace{0.5em}
\begin{lstlisting}[language=text,breaklines=true,frame=single]
Find the most relevant papers and write a short two paragraph lit review based on the "prose" context. Explain how our work adds to the literature by proposing a new way to think about the valuation of AI stocks.

Be sure to cite:
  - Korinek and Suh (2024)
  - Babina et al (2023) "Artificial Intelligence and Firms' Systematic Risk"
  - Zhang (2019) "Labor-Technology"

Respond with only new content. Do not repeat anything in the context.  

\end{lstlisting}
\vspace{1em}
\subsection*{Prompt 7: 07-conclusion-prose}
\textbf{Model:} sonnet
\vspace{0.5em}
\begin{lstlisting}[language=text,breaklines=true,frame=single]
Write a short "Conclusion" section. 

- Review the main argument 
  - Briefly touch on the formalization
- Investing in AI stocks may be an effective way to fund universal basic income (UBI)
  - Be cautious when suggesting this. 
  - It's dangerous to have government so involved in the economy.
  - Do not recommend a sovereign wealth fund.

Respond with only new content. Do not repeat anything in the context.  

\end{lstlisting}
\vspace{1em}
\subsection*{Prompt 8: 08-introduction-prose}
\textbf{Model:} sonnet
\vspace{0.5em}
\begin{lstlisting}[language=text,breaklines=true,frame=single]
Write the "Introduction" section, based on the "prose" context. 

Start with background. Describe how AI progress is happening quickly, and investors may be concerned about their wages being displaced.       

Then describe how technological change has occurred before, but AI is distinct because there is no product or service that AI could not, in principle, create.  An example is the current paper, which is entirely written by AI, using a few short prompts. Provide a link to the github site, which is https://github.com/chenandrewy/Prompts-to-Paper/. This differs from say, the internet revolution. AI progress may also be incredibly sudden (the AI singularity).

Then describe what the paper does. It studies how AI stocks are priced, given that there is the risk that AI may destroy livelihoods and consumption. 

End by incorporating the lit review (`06-litreview-notes`).

Include a footnote noting that "we" refers to one human author and multiple LLMs.

Cite papers as appropriate. Ensure citations correspond to items from bibtex-all.bib.      

\end{lstlisting}
\vspace{1em}
\subsection*{Prompt 9: 09-full-paper}
\textbf{Model:} sonnet
\vspace{0.5em}
\begin{lstlisting}[language=text,breaklines=true,frame=single]
Write a short paper titled "Hedging the AI Singularity" based on the "prose" context.

Omit the author and date.

Add an abstract of less than 100 words, not indented. The abstract should mention that the paper is written by AI.

Output a complete latex document, including preamble. `template.tex` as a template.

\end{lstlisting}
\vspace{1em}