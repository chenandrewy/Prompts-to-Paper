\documentclass{article}

\usepackage{amsfonts}
\usepackage{graphicx}
\usepackage{hyperref}
\usepackage[utf8]{inputenc}
\usepackage{textgreek}


% theorem environments
\usepackage{amsthm}  % Provides theorem environments
\usepackage{amsmath} % Enhanced math formatting
\usepackage{amssymb} % Additional mathematical symbols
\usepackage{thmtools} % Enhanced theorem formatting

% Basic theorem-like environments
\theoremstyle{plain}    % Used for theorems, propositions, lemmas (italicized text)
\newtheorem{theorem}{Theorem}[section]  % Numbered by section
\newtheorem{proposition}[theorem]{Proposition}  % Shares numbering with theorems
\newtheorem{lemma}[theorem]{Lemma}
\newtheorem{corollary}[theorem]{Corollary}

% Definition-like environments
\theoremstyle{definition}  % Normal text (not italicized)
\newtheorem{definition}[theorem]{Definition}  % Shares numbering with theorems
\newtheorem{example}[theorem]{Example}
\newtheorem{remark}[theorem]{Remark}

% Bibliography settings
\usepackage[backend=biber,style=authoryear,uniquename=false, maxbibnames=99, maxcitenames=3, uniquelist=false, natbib=true]{biblatex}
\addbibresource{../input-other/all-bib.bib}

\begin{document}

\title{Hedging the AI Singularity}
\author{}
\date{}

\maketitle

\begin{abstract}
This paper examines the financial implications of AI-induced labor displacement by modeling the AI singularity as a potential shock to the capital share of income. We develop a neoclassical growth model in which the capital share parameter may suddenly increase, representing the arrival of transformative AI technology that substantially shifts income from labor to capital. We characterize the price of capital in this environment and derive expressions for the risk premium associated with AI singularity risk. Crucially, we decompose this premium to isolate the hedging value of capital against potential labor income losses. Our analysis reveals that forward-looking investors might optimally adjust their portfolios to include AI-exposed assets as a hedge against the possibility of AI-induced wage displacement, even if these assets offer lower expected returns in normal times.
\end{abstract}

\section{Introduction}

The rapid advancement of artificial intelligence (AI) technologies has emerged as one of the most significant economic developments of the 21st century. Recent breakthroughs in large language models, computer vision, and reinforcement learning have dramatically expanded the scope of tasks that can be automated, raising concerns about the potential displacement of human labor across various sectors \citep{acemoglu2020}. As AI systems become increasingly capable of performing complex cognitive tasks once thought to be the exclusive domain of humans, investors face growing uncertainty about the future value of their human capital and labor income streams \citep{garleanu2012, kogan2020}.

Technological change has been a constant feature of economic history, from the Industrial Revolution to the digital transformation of recent decades. However, AI represents a fundamentally different kind of technological shift. Unlike previous innovations that automated specific physical tasks or enhanced particular industries, AI has the potential to impact virtually every sector of the economy simultaneously \citep{babina2024}. The internet revolution, for instance, primarily transformed information access and communication channels \citep{ofek2003, hong2006}, but still required extensive human involvement to create content and services. In contrast, there is theoretically no product or service that sufficiently advanced AI could not, in principle, create or deliver \citep{chen2024}.

Moreover, some researchers and technologists have proposed the possibility of an "AI singularity"—a hypothetical point at which AI systems become capable of recursive self-improvement, potentially leading to an intelligence explosion and rapid, discontinuous technological change \citep{gofman2024}. While the timing and likelihood of such an event remain subjects of debate, the possibility introduces a unique form of tail risk for labor markets and asset prices \citep{rietz1988, barro2006}. This risk is characterized not just by the magnitude of potential disruption but also by the suddenness with which it might occur.

This paper examines the financial implications of AI-induced labor displacement by modeling the AI singularity as a potential sudden shock to the capital share of income. We develop a neoclassical growth model in which the capital share parameter $\alpha$ may jump to a significantly higher value $\alpha'$ with a fixed probability $p$, representing the arrival of transformative AI technology that substantially increases the returns to capital relative to labor. This approach builds on the disaster risk literature \citep{gourio2013, wachter2013} but focuses specifically on the redistribution of income from labor to capital rather than aggregate output declines.

The remainder of the paper is organized as follows. Section 2 presents our theoretical model of AI singularity risk. Section 3 derives the asset pricing implications and characterizes the hedging value of capital against labor income loss. Section 4 examines the extreme case where $\alpha'$ approaches 1, representing near-complete automation. Section 5 concludes.

\section{A Neoclassical Growth Model with Capital Share Disaster Risk}

\subsection{Model Setup}

We consider a standard neoclassical growth model with a representative agent and a Cobb-Douglas production function. The key feature of this model is that the capital share parameter $\alpha$ can suddenly jump to a higher value $\alpha'$ with probability $p$ in each period, representing a "disaster" that shifts income from labor to capital.

The production function is given by:
\begin{equation}
Y_t = K_t^{\alpha_t} L_t^{1-\alpha_t}
\end{equation}

where $Y_t$ is output at time $t$, $K_t$ is the capital stock, $L_t$ is labor input (normalized to 1), and $\alpha_t$ is the capital share parameter, which follows:
\begin{equation}
\alpha_{t+1} = 
\begin{cases}
\alpha' & \text{with probability } p \\
\alpha & \text{with probability } 1-p
\end{cases}
\end{equation}
where $\alpha' > \alpha$. Once the capital share jumps to $\alpha'$, it remains at that value permanently, representing an irreversible technological transformation.

Capital evolves according to:
\begin{equation}
K_{t+1} = (1-\delta)K_t + I_t
\end{equation}
where $\delta$ is the depreciation rate and $I_t$ is investment.

\subsection{Household Problem}

The representative household maximizes expected lifetime utility:
\begin{equation}
\mathbb{E}_0 \sum_{t=0}^{\infty} \beta^t U(C_t)
\end{equation}

where $\beta$ is the discount factor and $U(C_t)$ is the utility function, which we assume takes the constant relative risk aversion (CRRA) form:
\begin{equation}
U(C_t) = \frac{C_t^{1-\gamma}}{1-\gamma}
\end{equation}
with $\gamma > 0$ being the coefficient of relative risk aversion.

The household's budget constraint is:
\begin{equation}
C_t + I_t = r_t K_t + w_t L_t
\end{equation}

where $r_t$ is the rental rate of capital and $w_t$ is the wage rate.

Under competitive markets, factor prices equal their marginal products:
\begin{align}
r_t &= \alpha_t K_t^{\alpha_t-1} L_t^{1-\alpha_t} = \alpha_t \frac{Y_t}{K_t} \\
w_t &= (1-\alpha_t) K_t^{\alpha_t} L_t^{-\alpha_t} = (1-\alpha_t) Y_t
\end{align}

\section{Asset Pricing Implications and Hedging Value}

\subsection{Euler Equation and Asset Pricing}

The household's optimization problem yields the standard Euler equation:
\begin{equation}
U'(C_t) = \beta \mathbb{E}_t[U'(C_{t+1})(1-\delta+r_{t+1})]
\end{equation}

This can be rewritten to determine the price of capital $P_t^K$:
\begin{equation}
P_t^K = \mathbb{E}_t\left[\frac{\beta U'(C_{t+1})}{U'(C_t)}(r_{t+1} + (1-\delta)P_{t+1}^K)\right]
\end{equation}

Let's denote the stochastic discount factor as $M_{t,t+1} = \frac{\beta U'(C_{t+1})}{U'(C_t)}$. The price of capital can be written as:

\begin{align}
P_t^K &= (1-p)\mathbb{E}_t[M_{t,t+1}(r_{t+1}(\alpha) + (1-\delta)P_{t+1}^K(\alpha)) | \alpha_{t+1}=\alpha] \\
&+ p\mathbb{E}_t[M_{t,t+1}(r_{t+1}(\alpha') + (1-\delta)P_{t+1}^K(\alpha')) | \alpha_{t+1}=\alpha']
\end{align}

\subsection{Characterizing the Hedging Value}

To highlight the hedging value, we express consumption in terms of wage income and capital income:

\begin{equation}
C_t = w_t L_t + r_t K_t - I_t = (1-\alpha_t)Y_t + \alpha_t Y_t - I_t = Y_t - I_t
\end{equation}

When $\alpha$ jumps to $\alpha'$ (assuming $\alpha' > \alpha$), the share of output going to labor $(1-\alpha_t)$ decreases, reducing wage income. However, the share going to capital increases, potentially providing a hedge against this labor income loss.

To explicitly characterize the hedging value, we can take a first-order approximation of the stochastic discount factor around the steady state and decompose the covariance terms:

\begin{align}
P_t^K &\approx \mathbb{E}_t[M_{t,t+1}]\mathbb{E}_t[r_{t+1} + (1-\delta)P_{t+1}^K] + \text{Cov}_t(M_{t,t+1}, r_{t+1} + (1-\delta)P_{t+1}^K)
\end{align}

The covariance term captures the hedging value. To see this more clearly, we can further decompose it:

\begin{align}
\text{Cov}_t(M_{t,t+1}, r_{t+1} + (1-\delta)P_{t+1}^K) &= \text{Cov}_t\left(\frac{\beta U'(C_{t+1})}{U'(C_t)}, r_{t+1} + (1-\delta)P_{t+1}^K\right)
\end{align}

When $\alpha$ jumps to $\alpha'$, labor income $(1-\alpha_t)Y_t$ decreases while capital income $\alpha_t Y_t$ increases. If the utility function exhibits sufficient risk aversion, the marginal utility $U'(C_{t+1})$ increases when consumption falls due to the labor income shock. Simultaneously, the return on capital $r_{t+1}$ increases due to the higher capital share.

This creates a positive covariance between the stochastic discount factor and capital returns, resulting in a hedging premium that increases the price of capital relative to its expected payoff. This is consistent with the findings of \citet{garleanu2012} and \citet{kogan2020}, who show that technological progress creates displacement risk that affects asset pricing by making stocks of innovative firms serve as hedges.

\subsection{Quantifying the Hedging Value}

To quantify this hedging value, we can express the wage in the state where $\alpha$ jumps to $\alpha'$:

\begin{equation}
w_{t+1}(\alpha') = (1-\alpha')K_{t+1}^{\alpha'}L_{t+1}^{-\alpha'}
\end{equation}

Compared to the wage if $\alpha$ had remained unchanged:

\begin{equation}
w_{t+1}(\alpha) = (1-\alpha)K_{t+1}^{\alpha}L_{t+1}^{-\alpha}
\end{equation}

The percentage loss in labor income due to the jump in capital share is approximately:

\begin{equation}
\frac{w_{t+1}(\alpha) - w_{t+1}(\alpha')}{w_{t+1}(\alpha)} \approx \frac{\alpha' - \alpha}{1-\alpha} + (\alpha' - \alpha)\ln\left(\frac{K_{t+1}}{L_{t+1}}\right)
\end{equation}

This expression shows that the labor income loss depends on both the magnitude of the capital share jump $(\alpha' - \alpha)$ and the capital-labor ratio. The hedging value of capital is higher when this potential labor income loss is larger.

The optimal portfolio choice will involve a significant hedging demand for assets that pay off well when $\alpha$ jumps to $\alpha'$. Such assets would provide insurance against the catastrophic decline in labor income. The hedging demand can be quantified as:

\begin{align}
\theta_H = \frac{\gamma \cdot \text{Cov}(R_{t+1}, \log(w_{t+1}/w_t))}{\sigma^2_R}
\end{align}

Where $\gamma$ is the coefficient of relative risk aversion, $R_{t+1}$ is the return on the hedging asset, and $\sigma^2_R$ is its variance. When $\alpha'$ is significantly larger than $\alpha$, this covariance becomes strongly negative, creating a large positive hedging demand for capital assets that benefit from the regime shift.

\section{The Extreme Case: Near-Complete Automation}

We now consider the extreme case where $\alpha'$ approaches 1, representing a scenario of near-complete automation where capital captures almost all economic output. This scenario resembles concerns about technological unemployment from advanced AI, where capital (including AI systems) could theoretically capture nearly all economic output.

\subsection{Human Capital Valuation}

The present value of human capital (PVHC) can be expressed as:
\begin{align}
\text{PVHC}_t = w_t + \sum_{j=1}^{\infty}E_t\left[\frac{w_{t+j}}{\prod_{i=1}^{j}(1+r_{t+i}-\delta)}\right]
\end{align}

When there's a probability $p$ of $\alpha$ jumping to $\alpha' \approx 1.0$, the expected present value becomes:
\begin{align}
\text{PVHC}_t &= w_t + (1-p)\sum_{j=1}^{\infty}E_t\left[\frac{w_{t+j}(s)}{\prod_{i=1}^{j}(1+r_{t+i}(s)-\delta)}\right] + p\sum_{j=1}^{\infty}E_t\left[\frac{w_{t+j}(s')}{\prod_{i=1}^{j}(1+r_{t+i}(s')-\delta)}\right]\\
&\approx w_t + (1-p)\sum_{j=1}^{\infty}E_t\left[\frac{w_{t+j}(s)}{\prod_{i=1}^{j}(1+r_{t+i}(s)-\delta)}\right]
\end{align}

The second term in the last line approaches zero because $w_{t+j}(s') \approx 0$ when $\alpha' \approx 1.0$. This represents a catastrophic decline in the value of human capital, which would have profound implications for household welfare and portfolio choice.

\subsection{Implications for Portfolio Choice}

In this extreme case, the hedging value of capital becomes paramount. Individuals would optimally hold significant capital assets as insurance against this scenario, even if these assets offer lower expected returns in normal times. This provides a rational explanation for why investors might accept seemingly low risk premiums on AI-related investments, consistent with the empirical findings of \citet{babina2024} and \citet{eisfeldt2023} that AI-exposed firms experience higher market valuations.

This hedging motive may help explain the high valuations of technology companies at the forefront of AI development. Similar to the patterns observed during the dot-com bubble \citep{ofek2003, hong2006}, investors may be willing to pay a premium for stocks that offer protection against labor income risk, even if the probability of an AI singularity is relatively small.

However, this behavior differs from the purely speculative dynamics documented by \citet{greenwood2009} and \citet{griffin2011} during the tech bubble. Rather than being driven solely by inexperienced investors or speculation, high valuations of AI-exposed firms may reflect a rational hedging demand against the tail risk of labor displacement.

\section{Conclusion}

In this paper, we have developed a neoclassical growth model with a stochastic capital share to analyze the financial implications of AI-induced labor displacement. Our key finding is that capital serves as a natural hedge against shifts in factor shares, providing insurance against the risk of labor income loss due to technological change.

The hedging value of capital is particularly significant in the extreme case where AI technology could potentially capture a very large share of economic output. In such scenarios, forward-looking investors would optimally hold AI-exposed assets as insurance against the potential collapse in labor income, even if these assets offer lower expected returns in normal times.

Our analysis provides a rational explanation for the high valuations of AI-exposed firms observed in the market. Rather than reflecting irrational exuberance, these valuations may incorporate a hedging premium against the tail risk of labor displacement. This perspective complements the empirical findings of \citet{babina2024} and \citet{cao2024} on the impact of AI on firm valuations and stock returns.

Future research could extend our model to incorporate heterogeneity in labor income risk across different occupations and industries, as well as explore the implications for wealth inequality and social welfare. Additionally, empirical work could test our predictions by examining the relationship between AI exposure, labor income risk, and asset returns.

\printbibliography

\end{document}