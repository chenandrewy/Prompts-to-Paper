\section{Prompts Used to Generate This Paper}

    Each prompt consists of context and instructions. The context consists of the responses to the previous prompts, and may include literature reviews (all AI generated). For writing tasks (using Claude 3.7 Sonnet), a system prompt is also included.
    
    For further details, see \url{{https://github.com/chenandrewy/Prompts-to-Paper/}}.
    
    The system prompt and instructions are listed below.
    
\vspace{-2ex}
\subsection*{System Prompt (model: claude-3-7-sonnet-20250219)}
\vspace{-1ex}
\begin{lstlisting}[language=text,breaklines=true,frame=single]
You are an asset pricing theorist who publishes in the top journals (Journal of Finance, Journal of Financial Economics, Review of Financial Studies). You think carefully with mathematics and check your work, step by step. 

Your team is writing a paper with the following main argument: the high valuations of AI stocks could be in part because they hedge against a negative AI singularity (an explosion of AI development that is devastating for the representative investor). This contrasts with the common view that AI valuations are high due to future earnings growth. Since the AI singularity is inherently unpredictable, the paper is more qualitative than quantitative. The goal is to just make this point elegantly.

Write in prose. No headings and no bullet points. But do use display math to highlight key assumptions. Cite papers using Author (Year) format.

Be conversational yet rigorous. Favor plain english. Be direct and concise. Remove text that does not add value.     

Be modest. Do not overclaim.

Format the math nicely. Use we / our / us to refer to the writing team.  

\end{lstlisting}
\vspace{-3ex}
\subsection*{Instruction: 01-model-prose  (model: claude-3-7-sonnet-20250219)}
\vspace{-1ex}
\begin{lstlisting}[language=text,breaklines=true,frame=single]
Draft the model description. The model is purposefully simple and captures the essence of the main argument. Only describe the assumptions. No results or insights.
  - Two agents
    - AI owners: Fully invested in AI, not marginal investors
    - Representative household: Marginal investor, only their consumption matters, CRRA
  - Representative household's gross consumption growth
    - is either 1 or e\\^(-b) (disaster)
      - A disaster is a revolutionary improvement in AI that is devastating for the household
      - Benefits of AI improvement are captured by the AI owners
      - For the household, labor income, way of life, meaning is lost 
      - At t=0, no disasters have happened (singularity has not occurred)
      - Multiple disasters may happen, capturing ongoing uncertainty if a singularity occurs
  - A publicly traded AI asset
    - Dividend is a small fraction of consumption before the singularity
    - Each time a disaster occurs, the dividend's fraction of consumption grows by a factor of e\\^h
    - Meant to capture a worst case scenario, where the dividend may actually shrink in each disaster
      - i.e. AI improvements are concentrated in privately-held AI assets

\end{lstlisting}
\vspace{-3ex}
\subsection*{Instruction: 02-result-notes  (model: o1)}
\vspace{-1ex}
\begin{lstlisting}[language=text,breaklines=true,frame=single]
Find the price/dividend ratio of the AI asset at t = 0. Show the derivation, step by step.

\end{lstlisting}
\vspace{-3ex}
\subsection*{Instruction: 03-table-notes  (model: o3-mini)}
\vspace{-1ex}
\begin{lstlisting}[language=text,breaklines=true,frame=single]
Make a table of the price/dividend for b from 0.40 to 0.95 and prob of disaster from 0.0001 to 0.02. Here, fix h = 0.20, CRRA = 2, time preference = 0.96. If the price is infinite, use "Inf". Round to the nearest whole number.

\end{lstlisting}
\vspace{-3ex}
\subsection*{Instruction: 04-resultandtable-prose  (model: claude-3-7-sonnet-20250219)}
\vspace{-1ex}
\begin{lstlisting}[language=text,breaklines=true,frame=single]
Convert the notes in `02-result-notes` and `03-table-notes` into prose. The prose is intended to immediately follow `01-model-prose` and should flow naturally. Include the table.

\end{lstlisting}
\vspace{-3ex}
\subsection*{Instruction: 05-litreview-prose  (model: claude-3-7-sonnet-20250219)}
\vspace{-1ex}
\begin{lstlisting}[language=text,breaklines=true,frame=single]
Find the most relevant papers and write a short two paragraph lit review based on the "prose-response" context. 
Be sure to cite:
  - Korinek and Suh (2024); Babina et al (2023) on "Systematic Risk"; Zhang (2019) "Labor-Technology"

\end{lstlisting}
\vspace{-3ex}
\subsection*{Instruction: 06-full-paper  (model: claude-3-7-sonnet-20250219)}
\vspace{-1ex}
\begin{lstlisting}[language=text,breaklines=true,frame=single]
Write a paper titled "Hedging the AI Singularity" based on the "prose-response" context.

Title page should have
- Title: "Hedging the AI Singularity"
- Authors: Andrew Y. Chen (Fed Board)
- Date: April 2025
- Abstract: Less then 100 words, mention at the end that the paper is generated by AI.

Main text should discuss
- Recent fast progress of AI (e.g. Chollet et al)
- Unlike previous technological revolutions, AI could create *any* good or service
  - This paper is generated by AI using six prompts (link to GitHub: https://github.com/chenandrewy/Prompts-to-Paper/)
  - Prompts are found in the appendix
  - Footnote: "we" refers to one human author and multiple LLMs
  - Raises concerns about labor income
- We are not saying a negative singularity will happen
  - But it is nevertheless important to consider this scenario        
- Related lit at end of Introduction
  - Unlike previous papers, this paper is generated by AI
- Derivation of the key formulas
  - It's OK to be verbose here because this shows the power of AI
- High price/dividend ratios, even though dividends never grow
- Key concept: AI assets provide partial (not complete) hedging against AI disaster risk
  - Variable 'b' represents net effect of (1) AI disaster and (2) AI asset dividend
  - Representative household cannot buy all AI assets (consistent with reality)
- Elaborate calibration is unreasonable given singularity uncertainty
- Contribution: novel perspective on AI stock valuation
- Universal Basic Income (UBI), but cautiously
  - Be centrist. Don't discuss redistribution nor free markets.

Text should avoid      
- Being overly academic
- Politics
  - Recommending sovereign wealth funds
  - Recommending heavy government involvement in the economy

Output a complete latex document, including preamble. `template.tex` as a template. Keep the appendix that is already in the template.

\end{lstlisting}
\vspace{-3ex}