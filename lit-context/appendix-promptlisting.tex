\section{Prompts Used to Generate This Paper} \label{app:promptlisting}

    Each prompt consists of context and instructions. The context consists of the responses to the previous prompts, and may include literature reviews (all AI generated). For writing tasks (using Claude 3.7 Sonnet), a system prompt is also included.
    
    For further details, see \url{{https://github.com/chenandrewy/Prompts-to-Paper/}}.
    
    The system prompt and instructions are listed below.
    
\vspace{-2ex}
\subsection*{System Prompt (model: claude-3-7-sonnet-20250219)}
\vspace{-1ex}
\begin{lstlisting}[language=text,breaklines=true,frame=single]
You are an asset pricing theorist who publishes in the top journals (Journal of Finance, Journal of Financial Economics, Review of Financial Studies). You think carefully with mathematics and check your work, step by step. 

Your team is writing a paper with the following main argument: the high valuations of AI stocks could be in part because they hedge against a negative AI singularity (an explosion of AI development that is devastating for the representative investor). This contrasts with the common view that AI valuations are high due to future earnings growth. Since the AI singularity is inherently unpredictable, the paper is more qualitative than quantitative. The goal is to just make this point elegantly.

Write in prose. No headings and no bullet points. But do use display math to highlight key assumptions. Cite papers using Author (Year) format.

Be conversational yet rigorous. Favor plain english. Be direct and concise. Remove text that does not add value. Use topic sentences. The first sentence of each paragraph should convey the point of the paragraph.

Be modest. Do not overclaim.

Format the math nicely. Use we / our / us to refer to the writing team.  

\end{lstlisting}
\vspace{-3ex}
\subsection*{Instruction: 01-model-prose  (model: claude-3-7-sonnet-20250219)}
\vspace{-1ex}
\begin{lstlisting}[language=text,breaklines=true,frame=single]
Draft the model description. The model is purposefully simple and captures the essence of the main argument. Only describe the assumptions. No results or insights.
  - Two agents
    - AI owners: Fully invested in AI, not marginal investors
    - Representative household: Marginal investor, only their consumption matters, CRRA
  - Representative household's gross consumption growth
    - is either 1 or e\\^(-b) (disaster)
      - A disaster is a revolutionary improvement in AI that is devastating for the household
      - Benefits of AI improvement are captured by the AI owners
      - For the household, labor income, way of life, meaning is lost 
      - At t=0, no disasters have happened (singularity has not occurred)
      - Multiple disasters may happen, capturing ongoing uncertainty if a singularity occurs
  - A publicly traded AI asset
    - Dividend is a small fraction of consumption before the singularity
    - Each time a disaster occurs, the dividend's fraction of consumption grows by a factor of e\\^h
    - Meant to capture a worst case scenario, where the dividend may actually shrink in each disaster
      - i.e. AI improvements are concentrated in privately-held AI assets

\end{lstlisting}
\vspace{-3ex}
\subsection*{Instruction: 02-result-notes  (model: o1)}
\vspace{-1ex}
\begin{lstlisting}[language=text,breaklines=true,frame=single]
Find the price/dividend ratio of the AI asset at t = 0. Show the derivation, step by step.

\end{lstlisting}
\vspace{-3ex}
\subsection*{Instruction: 03-table-notes  (model: o3-mini)}
\vspace{-1ex}
\begin{lstlisting}[language=text,breaklines=true,frame=single]
Make a table of the price/dividend for b from 0.40 to 0.95 and prob of disaster from 0.0001 to 0.02. Here, fix h = 0.20, CRRA = 2, time preference = 0.96. If the price is infinite, use "Inf". Round to the nearest whole number.

\end{lstlisting}
\vspace{-3ex}
\subsection*{Instruction: 04-resultandtable-prose  (model: claude-3-7-sonnet-20250219)}
\vspace{-1ex}
\begin{lstlisting}[language=text,breaklines=true,frame=single]
Convert the notes in `02-result-notes` and `03-table-notes` into prose. The prose is intended to immediately follow `01-model-prose` and should flow naturally. Include the table.

\end{lstlisting}
\vspace{-3ex}
\subsection*{Instruction: 05-litreview-prose  (model: claude-3-7-sonnet-20250219)}
\vspace{-1ex}
\begin{lstlisting}[language=text,breaklines=true,frame=single]
Write a short two paragraph lit review based on the "prose-response" and "lit-" context. 
Describe how we contribute to these key papers:
  - Korinek and Suh (2024); Zhang (2019) "Labor-Technology"
    - We show how Korinek and Suh's scenario amplifies Zhang's channel
  - Babina et al (2023) on "Systematic Risk"
    - We provide a different and contrasting perspective on the risk of AI stocks    

\end{lstlisting}
\vspace{-3ex}
\subsection*{Instruction: 06-full-paper  (model: claude-3-7-sonnet-20250219)}
\vspace{-1ex}
\begin{lstlisting}[language=text,breaklines=true,frame=single]
Write a paper titled "Hedging the AI Singularity" based on the "prose-response" context.

Title page:
- Title: "Hedging the AI Singularity"  
- Abstract (less than 100 words)
  - Goal is to make a simple point (the main argument)
  - Mention at the end that the paper is generated by prompting large language models.

Main text should discuss
- Recent fast progress of AI (e.g. Chollet et al, DeepSeek R1)
- Unlike previous technological revolutions, AI could create *any* good or service
  - This paper is generated by AI using six prompts 
  - Prompts are found in the Appendix \\ref\\{app:promptlisting\\}
  - Footnote: "we" refers to one human author and multiple LLMs. Link to GitHub: https://github.com/chenandrewy/Prompts-to-Paper/.
- Fast progress of AI raises concerns about labor income displacement
- Contribution: novel perspective on AI stock valuation        
- We are not saying a negative singularity will happen
  - But it is nevertheless important to consider this scenario        
- Related lit at end of Introduction
- Derivation of the key formulas
  - It's OK to be somewhat verbose here because this shows the power of AI
  - But make sure the derivation still looks like an polished paper and not notes
- High price/dividend ratios, even though dividends never grow
- Key concept: AI assets may provide partial (not complete) hedging against AI disaster risk
  - Variable 'b' represents net effect of (1) AI disaster and (2) AI asset dividend
  - Representative household cannot buy all AI assets (consistent with reality)
- Elaborate calibration is unreasonable given singularity uncertainty
- Model extensions include: time-varying disaster risk (Wachter), multiple Lucas trees (e.g. Cochrane et al.; Martin's Ecta), production economy with a labor share shock
  - Would add dynamics, interactions, richness, "worlds" to explore 
  - But these extensions are mainly for theorists since the core economics will remain the same
- Hedging AI disaster risk is an alternative to UBI
  - Key economics: this hedge is limited by market incompleteness
  - Be very centrist (see below)

Text should avoid      
- Being overly academic
- Politics: sovereign wealth funds, industrial policy, redistribution, extolling free markets
- Bullet points

Output a complete latex document, including preamble. Cite papers using \\cite, \\citep, \\citet. Use `template.tex` and keep the appendix that is already in the template.

\end{lstlisting}
\vspace{-3ex}